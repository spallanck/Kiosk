\section{Business Requirements}
\subsection{Background}
 The Western Washington University Computer Science Department inhabits the fourth floor of the Communications Building of Westerns Campus.  It contains staff, faculty and students enrolled in Computer Science classes.  It also contains a wealth of information from staff and faculty and staff for students, about upcoming classes, events and opportunities for the students to take advantage of.   The computer science department also contains a relatively unused touch screen monitor wired up to a raspberry pi, right in the entrance to the department offices. 

\subsection{Business Opportunity}
 The Western Washington University Computer Science Department has a lot of information that most, if not all, students would find useful to know.  However, information can be hard to spread.  It can be put up on flyers, or on bulletin boards, but that can often be drowned out by the sheer amount of information on display.  It can be put online, but even online information can be hard to find.  From a business standpoint, Western has nothing to gain financially with this project, but perhaps it can gain some expediency and ease to how information is sent out to students as well as accessibility of said information.

\subsection{Business Objectives and Success Criteria}
 Students pass by flyers everyday and don't notice them.  They can look at where a professor's office is on their phone, and still struggle to find its location.  Staff can send email after email, but some students will not even register that they are there beyond recognizing that it's from the department.  A new way to get information out to the students, in an eyecatching manner, could vastly help the student body in reaching goals set by faculty and themselves.  An increase in student response, or at least interest in some events, activities, classes, or even just higher office hour attendance would be a verifiable way to see success.

\subsection{Customer and Market Needs}
 The Western Washington Computer Science is putting this information out to the students, therefore it would behove the students to recognize it and registar the information.  A different route that leads to the same ending, the students being able to find offices, seeing flyers, seeing new classes, but in a way that they can act on.  The department wants this information to be accessible, so a new way to present it which would be accessible would benefit both the department students as well as the staff and faculty.


\subsection{Business Risks}
 Risk wise, the worst possible outcome is that nothing changes.  The students would still have flyers, bulletin boards and online information.  The staff would still put out information to the same accessibility and usability as before, and the student body would still respond with the same levels as before.  This outcome does not seem very likely, but there is always the possibility that the student body walks by our solution much like they do flyers in the halls.  Advertising the solution, and some of its features, may be the best way to avoid this possibility.
 

