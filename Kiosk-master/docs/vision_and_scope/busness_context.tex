
\section{Business Context}

\subsection{Stakeholder Profiles}
 Two stakeholder groups have been identified. Within these two groups there are sub categories with slightly different requirements to satisfy. The first group is the CS department, which is the primary stakeholder in this project. High priority requirements of the department are information accuracy, information availability, and information accessibility. The subcategories of the department include professors, academic support, front desk, information technical, and other faculty.
 Professors and academic support share requirements. As these two parties are often the focus of student enquiries ensuring they are easily identifiable from the rest of the department will be a requirement. Having more information about these individuals is another requirement to aid in student inquiries.
 Front desk and IT have requirements that overlap as well. As the project will be maintained by these parties a high priority requirement is making any maintenance either simple or unobtrusive to other tasks. Some examples are making the kiosk tamper resistant will aid IT so they don’t need to constantly keep tabs on it. Also ensuring that updating the displayed information doesn’t require hard to get access so that the front desk can modify any information as needed.
 Other faculty outside of the previously mentioned parties do not have any requirements that have not already been mentioned. Future iterations of this product may include them, but for this project they are not a major stakeholder.
 The second group of stakeholders is the student body. Subcategories include CS students, non CS students, clubs, and other visitors. High priority requirements are the same as the department's high priorities. As this is the main group that will be users of this product there are several requirements per party as well.
 For CS students the requirements closely mirror those of professors and academic support, as these are the groups that CS students will be interacting with the most. In addition to those requirements there are some that are shared for all students. The CS students are the main party in this group of stakeholders, so most other requirements will probably come from this group in future iterations. 
 Non CS students and visitors are also overlapping, as their goals with the kiosk are much different then the CS students. As such a requirement will be to have information displayed in a way anyone can understand, not just people who know the meanings of CS terms. Additionally having general purpose information will be a requirement of these groups as not to make them feel unincluded.
 The final stakeholder group is the CS clubs. While they do not have any additional requirements for this iteration, they may in a future version.


 

\subsection{Operating Environment}
 The operating environment of the Kiosk is simple, a large touch screen mounted on a wall. While simple it is still relevant to the design. As it will always be in the department we won’t need to drill down every single point of failure, as faculty will be there to reset it if ever needed. For the same reason we won’t need to worry about data accessibility, it will always have what it needs from the department. Being mounted on the wall is important for a few minor reasons. Buttons shouldn’t be placed in hard to reach sections like the very top of the screen. And as the user is standing, they should be able to get the information they need in just a few presses.
