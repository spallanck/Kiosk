\section{Vision of the Solution}
For Western Washington University Computer Science students and faculty who want to learn more about the Computer Science department and CS-related events, the Kiosk is an interactive touch screen located on the fourth floor of the Communications Facility. Unlike posters, emails, and the Western Washington University Computer Science webpage, our product aims to consolidate the information, news, and events pertaining to the Computer Science department and help students find the faculty members’ offices.

\subsection{Vision Statement}
 This is the formal vision statement.  
 \begin{tabular}{ll}
 For:         &user class\\
 Who:         &statement of need\\
 The:         &title of product\\
 Is:          &statement of solution\\
 Unlike:      &closest alternative solution\\
 Our Product: &differentiation statement  \\
 \end{tabular}  
\subsection{Major Features}
One major feature of the Kiosk is that it will include a map of the fourth floor of the CS department. This map will include the location of each faculty member’s office as well as the times that each faculty member is available in their office. Students will be able to search for a faculty member’s office by entering the faculty member’s name into a search bar to find out the location of the office and the times they can expect the faculty member to be there. Additionally, we plan to add information about each faculty member, including what classes they teach during the current quarter and their area of research.
 
The Kiosk will also have a slideshow that plays when it has been idling. After the screen has not been touched for a while, a slideshow of things pertaining to Western’s CS department will begin to play. This slideshow will include upcoming CS events, clubs, and projects developed by Western students. Information about an event will be added as the date of the event is nearing. Furthermore, each event will automatically be removed from the slideshow after the event has passed.


\subsection{Assumptions and Dependencies}
One assumption we are making is that faculty members will provide us with accurate schedules of their weekly routine. Part of our project hinges on students being able to know when they can find a faculty member in their office, so accuracy in the schedules we receive is crucial. This project also relies on the ability to discover upcoming events. Whether we collect our information about the events from Western’s CS webpage or from another source, knowing event details will be essential to maintaining the slideshow that plays as the Kiosk is idling.
 
Our project depends on the capabilities of the touch screen that users will interact with. Can the touch screen register a user’s finger sliding from one point to another? Can the touch screen register two taps in the same way a computer registers a double click on a mouse? Learning the functionality and limitations of the touch screen will shape the design choices we make as we develop the Kiosk. We will also depend on the CS department to provide us with a list of the faculty members and their office locations so that students are provided with the correct information if they are searching for a certain faculty member’s office.


